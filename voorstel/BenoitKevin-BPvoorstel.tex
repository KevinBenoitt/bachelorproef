%==============================================================================
% Sjabloon onderzoeksvoorstel bachproef
%==============================================================================
% Gebaseerd op document class `hogent-article'
% zie <https://github.com/HoGentTIN/latex-hogent-article>

% Voor een voorstel in het Engels: voeg de documentclass-optie [english] toe.
% Let op: kan enkel na toestemming van de bachelorproefcoördinator!
\documentclass{hogent-article}

% Invoegen bibliografiebestand
\addbibresource{voorstel.bib}

% Informatie over de opleiding, het vak en soort opdracht
\studyprogramme{Professionele bachelor toegepaste informatica}
\course{Bachelorproef}
\assignmenttype{Onderzoeksvoorstel}
% Voor een voorstel in het Engels, haal de volgende 3 regels uit commentaar
% \studyprogramme{Bachelor of applied information technology}
% \course{Bachelor thesis}
% \assignmenttype{Research proposal}

\academicyear{2023-2024} % TODO: pas het academiejaar aan

% TODO: Werktitel
\title{Adoptie en implementatie van Microsoft365 (M365) Governance Frameworks}

% TODO: Studentnaam en emailadres invullen
\author{Kevin Benoit}
\email{kevin.benoit@student.hogent.be}

% TODO: Medestudent
% Gaat het om een bachelorproef in samenwerking met een student in een andere
% opleiding? Geef dan de naam en emailadres hier
% \author{Yasmine Alaoui (naam opleiding)}
% \email{yasmine.alaoui@student.hogent.be}

% TODO: Geef de co-promotor op
\supervisor[Co-promotor]{S. Beekman (Synalco, \href{mailto:sigrid.beekman@synalco.be}{sigrid.beekman@synalco.be})}

% Binnen welke specialisatierichting uit 3TI situeert dit onderzoek zich?
% Kies uit deze lijst:
%
% - Mobile \& Enterprise development
% - AI \& Data Engineering
% - Functional \& Business Analysis
% - System \& Network Administrator
% - Mainframe Expert
% - Als het onderzoek niet past binnen een van deze domeinen specifieer je deze
%   zelf
%
\specialisation{Systeem \& Netwerk Administrator}
\keywords{M365, World Wide Web, $\lambda$-calculus}

\begin{document}
Dit onderzoeksvoorstel richt zich op het analyseren en verkennen van specifieke Governance Frameworks met een focus op National Institute of Standards and Technology (NIST), International Organization of Standardization (ISO) & Control Objectives for Information and Related Technologies (COBIT). De probleemstelling en centrale onderzoeksvraag zijn gericht op de impact, toepasbaarheid en voordelen dat deze frameworks bieden binnen IT-omgevingen met een focus op M365-implementaties. Het doel van dit onderzoek is om een inzicht te bieden voor de implementatie van deze frameworks binnen organisaties.  De methodologie bevat voornamelijk een grondige analyse van documentatie, interview met experts en een Proof of Concept (PoC) waar de verschillende frameworks zullen worden uitgewerkt. Er is ook een diepgaande literatuurstudie, waar er wordt onderzocht welke frameworks er bestaan, wat hun functie is en de compabiliteit binnen M365. 

\end{abstract}

\tableofcontents

% De hoofdtekst van het voorstel zit in een apart bestand, zodat het makkelijk
% kan opgenomen worden in de bijlagen van de bachelorproef zelf.
%---------- Inleiding ---------------------------------------------------------

\section{Introductie}%
\label{sec:introductie}

\subsection{Context}
Steeds meer organisaties maken de overstap naar M365 als centraal onderdeel van hun infrastructuur. Een gebrek aan duidelijke regelgeving en documentatie kan leiden tot beveiligingsrisico's, inconsistenties en verkeerd geïmplementeerde frameworks. Hierdoor is de behoefte aan duidelijk gestructureerde documentatie met betrekking tot Governance Frameworks een noodzaak.

\subsection{Onderzoek}
Het belang van Governance Frameworks binnen IT-omgevingen neemt steeds meer toe, vooral voor organisaties waar M365-implementaties centraal staan. Een geoptimaliseerde en beveiligde infrastructuur mag hierbij niet ontbreken. Dit onderzoeksvoorstel omvat een grondige analyse van specifieke Governance Frameworks, met de nadruk op NIST, ISO \& COBIT. 
Het doel van dit onderzoek is om antwoorden te vinden met betrekking tot de impact, toepasbaarheid en de voordelen die de frameworks zullen bieden. Ook wordt er gekeken naar welke uitdagingen en complicaties zich voordoen en hoe deze aangepakt moeten worden om een geoptimaliseerde infrastructuur te creëren. Ook zal

\subsection{Doelstelling}
De doelstelling van dit onderzoek is om diepgaande kennis en inzicht te verkrijgen over de verschillende Governance Frameworks, met name NIST, ISO \& COBIT. Op basis hiervan zullen richtlijnen en adviezen worden opgesteld voor de implementatie binnen IT-organisaties die gericht zijn op M365.

\subsection{Doelgroep}
De doelgroep voor dit onderzoek richt zich vooral op IT-consultants met een specialisatie in M365, informatiebeheerders en IT-professionals die verantwoordelijk zijn voor de werkwijze en implementatie van Governance-structuren. 

\subsection{Conclusie \& Eindresultaat}
Het resultaat van dit onderzoek zal afhangen van de toestandskoming van de documentatie of rapport betreft Governance frameworks binnen IT-omgevingen gericht op M365-implementaties. Hiervoor zal er dus ook een Proof of Concept (PoC) opgesteld worden om de frameworks te realiseren in een M365 tenant omgeving.


%---------- Stand van zaken ---------------------------------------------------

\section{Literatuurstudie}%
\label{sec:Literatuurstudie}

Er is een exponentiële groei van cloudgebaseerde oplossingen zoals M365, dit wordt steeds meer een volstrekt onderdeel binnen organisatieinfrastructuren. M365 biedt een veelheid aan tools die de coöperatie, effeciëntie \& productiviteit verhogen. Ook speelt de IT Governance een cruciale rol binnen organisaties. 
De compliance-vereisten en beveiligsstandaarden is een directe uitnodiging om Governance Frameworks te implementeren binnen organisaties die gericht zijn op M365.
In deze literatuurstudie gaan we nader ingaan op deze frameworks met name National Institute of Standards (NIST), International Organization of Standardization (ISO) \& Control Objectives for Information and Related Technologies (COBIT). Het onderzoek zal een analyse van voorgaande frameworks over de rol, toepasbaarheid of functionaliteit \& invloed binnen een organisatie.

\subsection{Governance Framework}
Governance frameworks verwijzen naar het geheel van processen, beleid en verantwoordelijkheden om de goede werking, privacy en beveiliging van IT-systemen binnen organisaties te garanderen. Het moet ervoor zorgen dat de werking zo effeciënt, risicovrij en regelgeving te waarborgen. Maar voornamelijk de vereisten en noden van de IT-organisatie te voldoen. Dit zijn de belangrijkste aspecten binnenin IT-Governance:

\begin{itemize}
  \item Risk Management
  Hier is het van belang om eventuele gevaren te identificeren, te bestuderen en op te lossen in verband met verschillende IT-processen
  \item Performance Management
  Dit is belangrijk om IT processen te evalueren en knelpunten te identificeren om zo verbetersmogelijkheden te vinden en een effectievere werkwijze op te stellen.
  \item Resource Management
  Het beheer van de IT resources, hier vallen ook de financiën, personeel \& infrastructuur in. Dit geeft de mogelijkheid aan de verantwoordelijken en bevoegden om ervoor te zorgen dat er continue ondersteuning beschikbaar is voor de huidige en toekomstige (onvoorziene) kosten en investeringen.
  \item Strategy
  Een grote portie van elke IT Governance implementatie vergt veel aandacht aan de strategie. Technologische vooruitgang is blijvende groeiend organisme. Een strategieplan bevat de prioriteiten die moeten uitgevoerd worden, een reeks aan initiatieven dat moeten gebeuren bij specifieke scenarios. Ook worden er Disaster Recovery Plans (DRP) uitgeschreven indien er een breuk in de infrastructuur is.
  \item Compliance \& Regulations
  Dit zijn de vereisten \& regelgevingen met betrekking tot privacy, beveiliging \& gegevensbescherming.
\end{itemize}

% Voor literatuurverwijzingen zijn er twee belangrijke commando's:
% \autocite{KEY} => (Auteur, jaartal) Gebruik dit als de naam van de auteur
%   geen onderdeel is van de zin.
% \textcite{KEY} => Auteur (jaartal)  Gebruik dit als de auteursnaam wel een
%   functie heeft in de zin (bv. ``Uit onderzoek door Doll & Hill (1954) bleek
%   ...'')



%---------- Methodologie ------------------------------------------------------
\section{Methodologie}%
\label{sec:methodologie}

Hier beschrijf je hoe je van plan bent het onderzoek te voeren. Welke onderzoekstechniek ga je toepassen om elk van je onderzoeksvragen te beantwoorden? Gebruik je hiervoor literatuurstudie, interviews met belanghebbenden (bv.~voor requirements-analyse), experimenten, simulaties, vergelijkende studie, risico-analyse, PoC, \ldots?

Valt je onderwerp onder één van de typische soorten bachelorproeven die besproken zijn in de lessen Research Methods (bv.\ vergelijkende studie of risico-analyse)? Zorg er dan ook voor dat we duidelijk de verschillende stappen terug vinden die we verwachten in dit soort onderzoek!

Vermijd onderzoekstechnieken die geen objectieve, meetbare resultaten kunnen opleveren. Enquêtes, bijvoorbeeld, zijn voor een bachelorproef informatica meestal \textbf{niet geschikt}. De antwoorden zijn eerder meningen dan feiten en in de praktijk blijkt het ook bijzonder moeilijk om voldoende respondenten te vinden. Studenten die een enquête willen voeren, hebben meestal ook geen goede definitie van de populatie, waardoor ook niet kan aangetoond worden dat eventuele resultaten representatief zijn.

Uit dit onderdeel moet duidelijk naar voor komen dat je bachelorproef ook technisch voldoen\-de diepgang zal bevatten. Het zou niet kloppen als een bachelorproef informatica ook door bv.\ een student marketing zou kunnen uitgevoerd worden.

Je beschrijft ook al welke tools (hardware, software, diensten, \ldots) je denkt hiervoor te gebruiken of te ontwikkelen.

Probeer ook een tijdschatting te maken. Hoe lang zal je met elke fase van je onderzoek bezig zijn en wat zijn de concrete \emph{deliverables} in elke fase?

%---------- Verwachte resultaten ----------------------------------------------
\section{Verwacht resultaat, conclusie}%
\label{sec:verwachte_resultaten}

Hier beschrijf je welke resultaten je verwacht. Als je metingen en simulaties uitvoert, kan je hier al mock-ups maken van de grafieken samen met de verwachte conclusies. Benoem zeker al je assen en de onderdelen van de grafiek die je gaat gebruiken. Dit zorgt ervoor dat je concreet weet welk soort data je moet verzamelen en hoe je die moet meten.

Wat heeft de doelgroep van je onderzoek aan het resultaat? Op welke manier zorgt jouw bachelorproef voor een meerwaarde?

Hier beschrijf je wat je verwacht uit je onderzoek, met de motivatie waarom. Het is \textbf{niet} erg indien uit je onderzoek andere resultaten en conclusies vloeien dan dat je hier beschrijft: het is dan juist interessant om te onderzoeken waarom jouw hypothesen niet overeenkomen met de resultaten.



\printbibliography[heading=bibintoc]

\end{document}