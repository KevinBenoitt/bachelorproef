%---------- Inleiding ---------------------------------------------------------

\section{Introductie}%
\label{sec:introductie}

\subsection{Context}
Steeds meer organisaties maken de overstap naar M365 als centraal onderdeel van hun infrastructuur. Een gebrek aan duidelijke regelgeving en documentatie kan leiden tot beveiligingsrisico's, inconsistenties en verkeerd geïmplementeerde frameworks. Hierdoor is de behoefte aan duidelijk gestructureerde documentatie met betrekking tot Governance Frameworks een noodzaak.

\subsection{Onderzoek}
Het belang van Governance Frameworks binnen IT-omgevingen neemt steeds meer toe, vooral voor organisaties waar M365-implementaties centraal staan. Een geoptimaliseerde en beveiligde infrastructuur mag hierbij niet ontbreken. Dit onderzoeksvoorstel omvat een grondige analyse van specifieke Governance Frameworks, met de nadruk op NIST, ISO \& COBIT. 
Het doel van dit onderzoek is om antwoorden te vinden met betrekking tot de impact, toepasbaarheid en de voordelen die de frameworks zullen bieden. Ook wordt er gekeken naar welke uitdagingen en complicaties zich voordoen en hoe deze aangepakt moeten worden om een geoptimaliseerde infrastructuur te creëren. Ook zal

\subsection{Doelstelling}
De doelstelling van dit onderzoek is om diepgaande kennis en inzicht te verkrijgen over de verschillende Governance Frameworks, met name NIST, ISO \& COBIT. Op basis hiervan zullen richtlijnen en adviezen worden opgesteld voor de implementatie binnen IT-organisaties die gericht zijn op M365.

\subsection{Doelgroep}
De doelgroep voor dit onderzoek richt zich vooral op IT-consultants met een specialisatie in M365, informatiebeheerders en IT-professionals die verantwoordelijk zijn voor de werkwijze en implementatie van Governance-structuren. 

\subsection{Conclusie \& Eindresultaat}
Het resultaat van dit onderzoek zal afhangen van de toestandskoming van de documentatie of rapport betreft Governance frameworks binnen IT-omgevingen gericht op M365-implementaties. Hiervoor zal er dus ook een Proof of Concept (PoC) opgesteld worden om de frameworks te realiseren in een M365 tenant omgeving.


%---------- Stand van zaken ---------------------------------------------------

\section{Literatuurstudie}%
\label{sec:Literatuurstudie}

Er is een exponentiële groei van cloudgebaseerde oplossingen zoals M365, dit wordt steeds meer een volstrekt onderdeel binnen organisatieinfrastructuren. M365 biedt een veelheid aan tools die de coöperatie, effeciëntie \& productiviteit verhogen. Ook speelt de IT Governance een cruciale rol binnen organisaties. 
De compliance-vereisten en beveiligsstandaarden is een directe uitnodiging om Governance Frameworks te implementeren binnen organisaties die gericht zijn op M365.
In deze literatuurstudie gaan we nader ingaan op deze frameworks met name National Institute of Standards (NIST), International Organization of Standardization (ISO) \& Control Objectives for Information and Related Technologies (COBIT). Het onderzoek zal een analyse van voorgaande frameworks over de rol, toepasbaarheid of functionaliteit \& invloed binnen een organisatie.

\subsection{Governance Framework}
Governance frameworks verwijzen naar het geheel van processen, beleid en verantwoordelijkheden om de goede werking, privacy en beveiliging van IT-systemen binnen organisaties te garanderen. Het moet ervoor zorgen dat de werking zo effeciënt, risicovrij en regelgeving te waarborgen. Maar voornamelijk de vereisten en noden van de IT-organisatie te voldoen. Dit zijn de belangrijkste aspecten binnenin IT-Governance:

\begin{itemize}
  \item Risk Management:
  Hier is het van belang om eventuele gevaren te identificeren, te bestuderen en op te lossen in verband met verschillende IT-processen
  \item Performance Management:
  Dit is belangrijk om IT processen te evalueren en knelpunten te identificeren om zo verbetersmogelijkheden te vinden en een effectievere werkwijze op te stellen.
  \item Resource Management:
  Het beheer van de IT resources, hier vallen ook de financiën, personeel \& infrastructuur in. Dit geeft de mogelijkheid aan de verantwoordelijken en bevoegden om ervoor te zorgen dat er continue ondersteuning beschikbaar is voor de huidige en toekomstige (onvoorziene) kosten en investeringen.
  \item Strategy:
  Een grote portie van elke IT Governance implementatie vergt veel aandacht aan de strategie. Technologische vooruitgang is blijvende groeiend organisme. Een strategieplan bevat de prioriteiten die moeten uitgevoerd worden, een reeks aan initiatieven dat moeten gebeuren bij specifieke scenarios. Ook worden er Disaster Recovery Plans (DRP) uitgeschreven indien er een breuk in de infrastructuur is.
  \item Compliance \& Regulations:
  Dit zijn de vereisten \& regelgevingen met betrekking tot privacy, beveiliging \& gegevensbescherming.
\end{itemize}

\subsection{ISO/IEC/IEEE}
De International Organization for Standardization (ISO) is een onafhankelijke, niet-gouvernementele organisatie die wereldwijd internationale normen vaststelt. Deze normen zijn bedoeld om de beveiliging en kwaliteit van IT-systemen te waarborgen en om een consistente en betrouwbare werking te garanderen.
Daarnaast is de International Electrotechnical Commission (IEC) verantwoordelijk voor de ontwikkeling van standaarden op het gebied van elektronische technologieën en verwante sectoren, waardoor uniforme normen en protocollen worden vastgesteld voor elektronica en aanverwante gebieden.
Het Institute of Electrical and Electronics Engineers (IEEE) richt zich op de vooruitgang van technologie, met een specifieke focus op elektronica, computertechnologieën en andere aanverwante disciplines. Hun bijdragen omvatten het vaststellen van standaarden, het publiceren van technische documenten en het organiseren van evenementen die de vooruitgang en standaardisatie binnen verschillende technologische gebieden bevorderen.
Er zijn natuurlijk verschillende ISO standaarden die in combinatie met de International Electrotechnical Commission (IEC) en Institute of Electrical and Electronics Engineers (IEEE) gaan samenwerken om zelf standaarden op te stellen. Enkele bekende ISO standaarden die ook van toepassing zijn voor dit onderzoek:

\begin{itemize}
  \item ISO/IEC 27001:
  Dit is een Information Systems Security Management (ISMS) Standard en heeft als doel het risico met betrekking tot informatiebeveiliging bij een minimum blijft. Het richt zich dus op het implementeren, onderhouden en het verbeteren van een ISMS \autocite{VladisLavV.2008}.
  \item ISO 31000:
  Dit is een standaard dat instaat voor het identificeren, beoordelen en controleren van de risicos. Zo worden de kans op risicos beperkt tot een minimum.
  Dit is een algemene standaard dat niet specifiek gericht is op IT-infrastructuren maar zeker een bijdrage kunnen leveren aan IT-omgevingen om ook zo risicos binnen M365-implementaties tot een minimum te behouden.
  \item ISO/IEC 38500:
  Ook deze standaard is van groot belang bij IT-infrastructuren. Dit staat in voor het belang van het IT-bestuur binnen organisaties. Deze standaard heeft als doel om de richtlijnen en regelgevingen te bezorgen aan het IT-bestuur van organisaties. Zo kan men beslissingen nemen en de compliance waarborgen.

\end{itemize}

\subsection{NIST}
Het National Institute of Standards and Technology (NIST) biedt een hoop richtlijnen en standaarden aan die een noodzaak worden voor de beveiliging en managament van IT-infrastructuren. De frameworks van NIST bieden een zeer grote bijdrage aan de beveiliging, cybersecurity, informatiebeveiliging en risico managament voor governance binnen M365-implementaties.
NIST zal dus een bijdrage leveren aan een optimale compliance- en beveiligingsstandaarden.
De Special Publication 800-171 standaard heeft als doel om Controlled Unclassified Information (CUI) te beschermen, elke organisatie heeft informatie die niet bedoeld is voor het publieke oog en moet dus beveiligd worden. Dit biedt de nodige richtlijnen om gevoelige gegevens te beschermen in IT-omgevingen.

\subsection{COBIT}
Control Objectives for Information and Related Technologies is ontwikkeld door Information Systems Audit and Control Association (ISACA) en biedt een structueele aanpak om IT-governance te begrijpen, evalueren, implementeren en beheren. COBIT biedt binnen dit onderzoek vooral hulp bij IT controls, optimaliseren van IT performances en voornamelijk voldoen aan de vereisten van de organisatie.
Hoewel COBIT geen specifiek framework heeft voor IT-infrastructuren, bieden de principes, regelgevingen, richtlijnen best practices van COBIT een cruciale rol. 

% Voor literatuurverwijzingen zijn er twee belangrijke commando's:
% \autocite{KEY} => (Auteur, jaartal) Gebruik dit als de naam van de auteur
%   geen onderdeel is van de zin.
% \textcite{KEY} => Auteur (jaartal)  Gebruik dit als de auteursnaam wel een
%   functie heeft in de zin (bv. ``Uit onderzoek door Doll & Hill (1954) bleek
%   ...'')



%---------- Methodologie ------------------------------------------------------
\section{Methodologie}%
\label{sec:methodologie}

\subsection{Literatuurstudie \& documentatieanalyse (3 weken)}
In eerste instantie zal er een literatuurstudie worden uitgevoerd om een beter begrip te krijgen over Governance Frameworks in organisaties in relatie tot M365-implementaties in hun infrastructuur. Er wordt diepgaand ingegaan op de verschillende soorten framworks en wat hun functie en bijdrage is. Ook zal er een uitgebreide documentatieanalyse voorkomen om inzichten te verkrijgen over eerdere case studies \& documentatie.

\subsection{Expertselectie \& Protocolontwikkeling (2 weken)}
Hier zal duidelijk gemaakt worden welke relevante experts worden geïnterviewed in IT Governance en M365 met een duidelijke criteria waarom ze relevante inbreng hebben op het onderzoek. Voor de protocolontwikkeling zal er een duidelijke en gedetailleerde interviewprotocol opgesteld worden om uiteindelijk een beter begrip te hebben over de werkelijke werkwijze en implementatie van IT Governance.
Het interviewprotocol zal bestaan uit relevante en specifiek gerichte vragen, richtlijnen en methoden bevatten om zo zelf relevante informatie te verkrijgen en te verwerken.

\subsection{Expertinterviews (2 weken)}
Na de expertselectie en protocolontwikkeling zal het feitelijke interview aangevangen worden. Het hoofddoel van de interview is om een informatie te verzamelen over IT Governance Frameworks gerelateerd aan M365-implementaties. In deze gesprekken zal er dieper ingegaan worden op de impact, werkwijze, toepasbaarheid en efficiëntie van de frameworks.
Er wordt gestreven om inzicht te krijgen in zowel de theoretische \& praktiche kennis van de gekozen IT-experts.

\subsection{Proof of Concept (PoC) (3 weken)}
Als derde fase zal er een een PoC worden uitgevoerd in een M365 tenant. Er zal gekeken worden naar:
\begin{itemize}
  \item Policy Implementation:
  Dit wordt gedaan aan de hand van richtlijnen en policies op te stellen binnen de tenant. Dit zal betrekking hebben tot informatiebeveiliging, toegangsbeheer \& overige relevante aspecten.
  \item User Management:
  Er zal diepgaand gekeken worden naar de verschillende mogelijkheden met gebruikersrollen, autorisatie \& toegangscontroles.
  \item Security \& Compliance:
  De aanpak hierbij zal bestaan uit het testen van de ingebouwde security functies van M365. Er wordt gekeken naar databeveiliging, logboekregistratie.
  \item Best Practices:
  Als laatste zal er gekeken worden wat de best practices zijn voor IT governance tools in een M365 omgeving. Met focus op NIST, ISO \& COBIT standaarden. 
\end{itemize}
Het doel van de PoC is om een actueel begrip te krijgen over de verschillende experimenten,  tools, functionaliteiten \& mogelijkheden  en hoe dit in praktijk kan worden opgesteld. 

\subsection{ Analyse \& Rapportage(3 weken)}
Als laatste fase zal er een uitbundige analyse worden uitgevoerd over de Expertinterviews \& PoC en hoe dit valt te vergelijken met de eerdere literatuurstudie. Alle bevindingen zullen dan in een rapportage worden opgesteld in combinatie met de literatuurstudie, expertise van IT-deskundigen en de PoC.


%---------- Verwachte resultaten ----------------------------------------------
\section{Verwacht resultaat, conclusie}%
\label{sec:verwachte_resultaten}

Hier beschrijf je welke resultaten je verwacht. Als je metingen en simulaties uitvoert, kan je hier al mock-ups maken van de grafieken samen met de verwachte conclusies. Benoem zeker al je assen en de onderdelen van de grafiek die je gaat gebruiken. Dit zorgt ervoor dat je concreet weet welk soort data je moet verzamelen en hoe je die moet meten.

Wat heeft de doelgroep van je onderzoek aan het resultaat? Op welke manier zorgt jouw bachelorproef voor een meerwaarde?

Hier beschrijf je wat je verwacht uit je onderzoek, met de motivatie waarom. Het is \textbf{niet} erg indien uit je onderzoek andere resultaten en conclusies vloeien dan dat je hier beschrijft: het is dan juist interessant om te onderzoeken waarom jouw hypothesen niet overeenkomen met de resultaten.

