\chapter{\IfLanguageName{dutch}{Stand van zaken}{State of the art}}
\label{ch:stand-van-zaken}

% Tip: Begin elk hoofdstuk met een paragraaf inleiding die beschrijft hoe
% dit hoofdstuk past binnen het geheel van de bachelorproef. Geef in het
% bijzonder aan wat de link is met het vorige en volgende hoofdstuk.

% Pas na deze inleidende paragraaf komt de eerste sectiehoofding.

 % Dit hoofdstuk bevat je literatuurstudie. De inhoud gaat verder op de inleiding, maar zal het onderwerp van de bachelorproef *diepgaand* uitspitten. De bedoeling is dat de lezer na lezing van dit hoofdstuk helemaal op de hoogte is van de huidige stand van zaken (state-of-the-art) in het onderzoeksdomein. Iemand die niet vertrouwd is met het onderwerp, weet nu voldoende om de rest van het verhaal te kunnen volgen, zonder dat die er nog andere informatie moet over opzoeken \autocite{Pollefliet2011}.

 %M Je verwijst bij elke bewering die je doet, vakterm die je introduceert, enz.\ naar je bronnen. In \LaTeX{} kan dat met het commando \texttt{$\backslash${textcite\{\}}} of \texttt{$\backslash${autocite\{\}}}. Als argument van het commando geef je de ``sleutel'' van een ``record'' in een bibliografische databank in het Bib\LaTeX{}-formaat (een tekstbestand). Als je expliciet naar de auteur verwijst in de zin (narratieve referentie), gebruik je \texttt{$\backslash${}textcite\{\}}. Soms is de auteursnaam niet expliciet een onderdeel van de zin, dan gebruik je \texttt{$\backslash${}autocite\{\}} (referentie tussen haakjes). Dit gebruik je bv.~bij een citaat, of om in het bijschrift van een overgenomen afbeelding, broncode, tabel, enz. te verwijzen naar de bron. In de volgende paragraaf een voorbeeld van elk.

 % \textcite{Knuth1998} schreef een van de standaardwerken over sorteer- en zoekalgoritmen. Experten zijn het erover eens dat cloud computing een interessante opportuniteit vormen, zowel voor gebruikers als voor dienstverleners op vlak van informatietechnologie~\autocite{Creeger2009}.

 %Let er ook op: het \texttt{cite}-commando voor de punt, dus binnen de zin. Je verwijst meteen naar een bron in de eerste zin die erop gebaseerd is, dus niet pas op het einde van een paragraaf.


Om de titel en onderzoek te verduidelijken, wordt er eerst onderzoek verricht naar verschillende relevante aspecten van het onderwerp. Onderdelen zoals Microsoft Purview, Retention labels, Sensitivity labels en Microsoft SharePoint alsook verscheidene standaarden zoals NIST en COBIT komen ter sprake.


\subsection{Governane Frameworks}

De digitale wereld blijft groeien waardoor Information Technology een belangrijkere rol begint te spelen in organisaties; Specifieker, organisaties die Microsoft365 implementeren in hun infrastructuur.
IT governance is het proces waarbij beslissingen worden gemaakt rond IT investments. Elke organisatie heeft wel een level van IT governance verwerkt, maar vaak is dit niet volledig gestructureerd en ontbreken er policies en is het heel informeel. \autocite{Symons2005}
Microsoft365 Governance Frameworks zijn dus gericht op policies, guidelines en best practices voor een bedrijf die M365 implementeert. De bedoeling hiervan is het waarborgen van efficiëntie van M365 services maar terwijl ook de veiligheid en compliance van de organisatie te garanderen.


\subsection{Azure Cloud Governance}

Het basis principe van cloud computing is dat een bedrijf in plaats van eigen hardware aan te kopen en beheren, dat ze resources huren van een cloud provider. Met als voordeel dat je alleen betaalt, wat je verbruikt. \autocite{Mikkola2021}
De cloud computing market heeft een grote sprong gemaakt in de IT wereld tijdens de Covid-19 pandemie, de combinatie met IT en remote werk zorgde voor een grote sprong in deze markt.
Natuurlijk zijn er verschillende Cloud Computing Deployment Models:

\begin{itemize}
    \item Private cloud:
    Dit is een cloud infrastructuur gemaakt voor één organisatie en niet meer. Dit kan dus on-premise beheerd worden of remote.
    Een reden om voor Private cloud te kiezen, is dat een eenvoudige manier is om gevoelige data en informatie te beveiligen en isoleren. \autocite{Mikkola2021}
    \item Public cloud:
    Dit is de cloud infrastructuur gemaakt voor publiek gebruik en wordt beheerd on premise door de Cloud provider.
    \item Hybrid cloud:
    Dit is de combinatie van de 2 voorgaande deployment models. Verschillende onderdelen binnen deze infrastuctuur zijn privaat, waar anderen dan weer publiekelijk zijn. Dit hangt volledig af van de voorafgeschreven scenario's.
\end{itemize}

Natuurlijk vraagt men zich af wat de voordelen nu zijn van Cloud Computing. Door de immense groei van toename in gebruikers van cloud computing, geeft dit ook een teken dat hier voordelen aan gekoppeld zijn.

\begin{itemize}
    \item Scalability:
    De mogelijkheid hebben om je resources te scalen gebaseerd op gebruik, verbruik en vraag. \autocite{Mikkola2021}
    \item Security:
    De security van de cloud infrastructuur is up-to-date en wordt regelmatig geüpdate naargelang de meest recente security threats.
    \item Geen overnodige on-premise hardware:
    De overbodige on-premise hardware wordt vervangen door een cloud dat remote te besturen is, dit is financieel heel voordelig.
    \item Return of Investment (ROI):
    Doordat een organisatie zelf in handen hebben, hoeveel er verbruikt kan worden binnenin hun organisatie. Hebben ze ook meer in hand wat de effectieve ROI zou zijn. Dit biedt een vooruitzicht op financiele doeleinden.
\end{itemize}


\subsection{ISO/IEC}

De International Organization for Standardization (ISO) is een onafhankelijk wereldwijd bedrijf dat instaat voor de normen waar een organisatie aan moet voldoen. Zo kan een bedrijf de veiligheid en kwaliteit van hun services garanderen.
ISO werkt nauw samen met International Electrotechnical Commission (IEC) dat verantwoordelijk is voor de ontwikkeling van standaarden op het gebied van elektronische technologieën en verwante sectoren, waardoor uniforme normen en protocollen worden vastgesteld voor elektronica en aanverwante gebieden. \autocite{Florea2016}
Natuurlijk zijn er verschillende ISO protocollen die werden aangemaakt om zo de normen te bereiken dat een bedrijf eist:

\begin{itemize}
    \item ISO/IEC 38500:
    Dit is een internationale standaard voor corporate governance of information technology. Het doel van deze standaard is om de organisatie management te voorzien van principles, definitions en een deployment model zodat de organisatie de Information Technology kan evalueren, beheren en monitoren. \autocite{Mikkola2021}
    \item ISO/IEC 27001:
    Dit is een standaard met sprake op Information Systems Security Management (ISMS) en heeft betrekking op Information Security en het aanliggende risico dat dit teweeg brengt. Het einddoel is de ISMS te optimaliseren op elk aspect. \autocite{AlMayahi2012}
    Dit wil dus zeggen dat het primaire doel vooral gericht is op vertrouwelijkheid, integriteit en beschikbaarheid van informatie. Hoe men dit bereikt is door het identificeren en beheersen van risico's die mogelijks invloed hebben op informatiebeveiliging.
    Dit helpt organisaties dus met het volgende:
    \begin{itemize}
        \item Ongeautoriseerde toegang tegenhouden en informatie beschermen
        \item Nauwkeurigheid en volledigheid van informatie te waarborgen
        \item Beschikbaarheid van de informatie garanderen
        \item De compliance en wetgevingen volgen
        \item Vertrouwen bij klanten opbouwen en behouden
    \end{itemize}

    De ISO 27001 is volgens de High-Level Structure (HLS) opgebouwd.
    
    \item ISO 31000:
    Dit heeft betrekking tot de Risk Management. Elke organisatie heeft wel een gradatie in hoe ze omgaan met Risk Management. Maar deze standaard heeft een talrijk aantal principles om de Risk Management te optimaliseren. \autocite{Florea2016}
    Deze standaard is niet specifiek gericht naar IT infrastructuren, maar kan wel een bijdrage leveren in elke organisatie. 
\end{itemize}


\subsection{ITIL}

ITIL werd origineel opgericht in de United Kingdom door de Central Computer and Telecommunications Agency (CCTA) in 1980 om de IT Service Management te verbeteren in de UK central government. \autocite{Potgieter2005}


De Information Technology Infrastructure Library (ITIL) is een wereldwijd gekend IT service management framework dat bestaat uit guidelines en best practices.  \autocite{Potgieter2005}
De framework zorgt ervoor dat er duidelijke structuur zit, om zo organisaties te helpen bij het leveren en beheren van hun IT en digital services. Om zo een een optimale value te leveren aan eindgebruikers, klanten en stakeholders. 
De ITIL framework laat de service provider toe om de services aan te passen naargelang de eisen, visie, strategieën en einddoel van een organisatie. \autocite{Mikkola2021}

De huidige versie van ITIL, ITIL 4, is gepubliceerd in Februari 2019. \autocite{Mikkola2021}
Natuurlijk vraagt men zich af, wat de relatie tot IT Governance dit brengt.De IT service management is gefocused op IT services en gerelateerde services.
Waar IT Governance dan vooral gefocust is op het enablen, controleren en assisteren bij het maken van beslissingen op strategisch niveau in organisaties. ITIL framework helpt hierbij op verschillende aspecten door het controleren en beheren van technologische veranderingen om zo het gebruik van IT services te verbeteren binnen het bedrijf. \autocite{Mikkola2021}

\subsection{COBIT}
Control Objectives for Information and Related Technologies is ontwikkeld door Information Systems Audit and Control Association (ISACA) en is een IT Governance Framework. Het doel van deze framework is om goede IT governance, technology management en de eventuele risk management te combineren. \autocite{Khther2013} 
Het is onafgebroken bezig met het updaten van policies en guidelines, maar werkt ook harmonisch samen met andere standaarden en guidelines. COBIT framework assisteert organisaties in de huidige challenges aan te pakken op vlak van de business arena door te relateren naar de requirements van businessen.
De meest recente versie van COBIT is COBIT 2019. De voorganger hiervan was COBIT 5, door gebrek aan support voor nieuwe technologie binnen de IT was de vraag groot naar een nieuwere versie.

Volgens ISACA biedt COBIT 2019 de volgende verbeteringen aan:

\begin{itemize}
    \item Flexibility
    Door de introductie van design factors kon COBIT 2019 flexibeler zijn en opener toe naar organisaties en gebruikers. Dit liet COBIT toe om aangepast te zijn aan specifieke user contexten. De COBIT open architecture geeft de toevoeging om meer te focussen op nieuwe areas en de modificaties op bestaane modellen zonder een impact teweeg te brengen op de core model van COBIT. \autocite{Mikkola2021}
    \item Relevantie
    Het COBIT Model ondersteunt alignment naar concepten toe die oorspronkelijk afstammen van een externe sources met name de laatste IT standaarden en compliance regulaties. \autocite{Khther2013}
    \item Performance Management of IT 
    De structuur van COBIT 2019 perfomance management model is geïntegreerd in het conceptuele model.
    \item Efficiëntie
    De efficiëntie van hoe information moet verwerkt worden op de meest optimale gebruik van resources.
    \item Confidentiality
    De bescherming van gevoelige informatie voor onbevoegde toegangsbeheer.
    \item Compliance
    De compliance met de wet, regulaties en contractuele arrangementen waar een business onderdeel van is.
\end{itemize}

Een onderdeel van COBIT is de COBIT Control Objectives. Deze bestaan om te assisteren in het bouwen van een pasbare management en control system binnen een IT omgeving. \autocite{Khther2013} 
Om ervoor te zorgen dat er een continue service voorzien is, dat gevestigd kan worden door de implementatie van meerdere controle procedures zoals het testen en schrijven van continuity plans. \autocite{Khther2013} 


\subsection{NIST}
National Institute of Standards and Technology (NIST) is opgericht door de Federal Chief Information Officier (CIO) Vivek Kundra om de federale overheid hun security adoptie te doen verbeteren en versnellen in verband met Cloud Computing.
Dit wordt gedaan door een talrijk aantal standaarden en guidelines op te richten in dicht verband met stakeholders, grote en private bedrijven. \autocite{Hogan2011}
Dit is ontstaan door voorgaande cyber incidenten. In Maart 2018 heeft de FBI een alert uitgestuurd omtrent een hevige groei in criminele activiteiten, vooral gericht op organisaties binnen de Energie sector.
Een meer recente cybersecurity aanval was die van in 2017 bij het bedrijf Equifax, criminele hackers hadden data gebreached van over de 147 millioen mensen over de hele wereld door security lekken en bad security practises uit te buiten. Dit heeft het bedrijf meer dan 240 millioen gekost. \autocite{Calder2018}

\begin{itemize}
     \item Cybersecurity en Information Technology
     NIST speelt een cruciale rol in de ontwikkeling van guidelines en standaarden omtrent Cybersecurity. Dit heet de NIST Cybersecurity Framework (CSF).
     Dit framework heeft 5 steeds voorkomende functies: Identify, Protect, Detect, Respond and Recover. \autocite{Ibrahim2018}
     Als deze 5 functies collectief worden geïmplementeerd in een bedrijf, geeft dit een goed beeld weer van hoe een high-end security standaard er zou moeten uitzien.
     Dit dient als een leiddraad voor bedrijven of organisaties om hun cybersecurity strategieplan te ontwikkelen en te verbeteren. \autocite{Almuhammadi2017}
     De framework biedt dus een gemeenschappelijke taalbegrip voor zowel interne als externe stakeholders. Het kan gebruikt worden voor:

     \begin{itemize}
        \item Het identificeren en prioritiseren van acties om de cybersecurity risk te verlagen
        \item Is een tool om de business, policy en technologische benadering te laten afstemmen met elkaar. \autocite{Calder2018}
        \item Om risico's te beheren voor de gehele organisatie.
        \item Om te focussen op kritische aspecten binnen een organisatie.
     \end{itemize}

    \item International
    NIST heeft een grote impact op de internationale gemeenschap rond beveiliging en standaarden. Veel organisaties hun beleid is gebaseerd op de NIST richtlijnen en gebruiken dit als referentiekader.
    Niet alleen de U.S. maakt gebruik van NIST. Landen zoals Italië en Uruguay hebben de framework reeds geïmplementeerd en andere landen hebben hun eigen variatie met NIST als hun referentiekader opgesteld. \autocite{Calder2018}
\end{itemize}

Om een beter begrip te krijgen hoe je data en informatie beveiligd moet je weten wat beveiiging eigenlijk betekent. Informatie en data moeten beveiligd volgens de CIA-driehoek: \autocite{Calder2018}
\begin{itemize}
    \item Confidentiality
    Informatie en data mogen alleen beschikbaar zijn voor mensen die het toegang verachten.
    \item Integrity
    Informatie en data zouden beschermd moeten zijn tegen data verlies of data verwijdering.
    \item Availability
    Informatie en data beschikbaarheid alleen voor bevoegden en bepaalde tijdstippen
\end{itemize}

\subsection{NIST Cybersecurity Framework (CSF)}

Doorgaans is NIST CSF implementeerbaar in elk soort organisatie. Maar is vooral gefocused op organisatie met kritieke data en informatie en een hoog risico lopen voor hackers.
De framework is een lopend document dat altijd updates en verbeteringen zal toevoegen met feedback van reeds bestaande implementaties. \autocite{Calder2018}
De CSF kan voor verschillende zaken worden gebruikt, het kan gebruikt worden om een geheel nieuw cybersecurity programma/plan op te zetten, om een bestaand programma te verbeteren en aan te passen. Maar kan ook een inzicht bieden op de huidige best practices van een bedrijf en hoe ze dit kunnen meenemen om het te verbeteren.
Doordat de Framework implementeerbaar is naar de noden van de organisatie en dit heel specifiek gericht kan zijn op de situatie van de organisatie. Zo kunnen ze heel kostgericht tewerk gaan en een optimale Return of Investment(ROI) creëren op vlak van cybersecurity.
Het biedt ook een algemene taal aan dat verstaanbaar is voor zowel technische mensen als business mensen. Zo is er zo goed als geen verwarring over wat de implementatie eigenlijk inhoudt en wat contracten, best practices, third-party documenten eigenlijk inhouden.
Het werkt heel nauw samen met de meest recente compliance requirements, doordat het bestaat uit best-practices en verscheidene sources en zijn eigen cybersecurity framework. Zo kan het voldoen aan de huidige standaarden, regel- en wetgevingen. \autocite{Calder2018}

De core van de framework bestaat uit verschillende cybersecurity activities. Deze activiteiten zijn gegroepeert in /Subcategories die zelf gegroepeerd zijn in /Categories. Deze categoriën zijn opgedeeld in 5 functies die eerder al ter sprake kwamen.
Namelijk:

\begin{itemize}
    \item Identify
    \item Protect
    \item Detect
    \item Respond
    \item Recover
\end{itemize}

De framework profile dat te vinden is in /profile, is een tool om de cybersecurity postuur te documenteren en te implementeren; Om zo de cybersecurity te verbeteren binnen een organisatie.
Dit bevat de huidige staat van de cybersecurity binnen een bedrijf maar ook de toekomstige plannen om eventuele gevaren weg te werken. \autocite{Sultan2017}
Een andere tool is bijvoorbeeld de Tiers tool van NIST CSF. Dit dient als een visionary tool dat een organisatie helpt om te begrijpen welke risico's het bedrijf nog oploopt en hoe ze dit kunnen benaderen, alsook welke huidige processen er zijn om de risico's te managen.
Deze processen worden dan geïdentificeerd en geclassificeerd in verscheidene Tiers: \autocite{Sultan2017}

\begin{itemize}
    \item Tier 1 Partial
    \item Tier 2 Risk informed
    \item Tier 3 Repeatable
    \item Tier 4 Adaptive
\end{itemize} 

De compliance met andere standaarden blijft ook heel belangrijk voor de werking. Er zijn enkele gebrekken die NIST CISF naar boven haalt als het gaat over compliance met de volgende standaarden.
De Monitor, Evaluate and Assess Compliance with External Requirements (MEA03) is een COBIT proces dat dus niet gemapped kan worden in de NIST CSF.
Ook ISO/IEC 27001 heeft ook een proces (A.18), dat maar gedeeltelijk gemapped is naar NIST CSF toe.



\subsection{Zero Trust Architecture (ZTA)}

Een Zero Trust Architecture is een cybersecurity architectuur dat gebaseerd is op zero trust principles om data breuken en interne data wisseling te voorkomen en limiteren.
Dit wordt steeds belangrijker in de infrastructuur van organisaties. Zero Trust is geen enkele architectuur, maar een set van verscheidene guiding principles voor de workflow, standaarden, systeem designs en operations dat uiteindelijk gebruikt kunnen worden om de security in een organisatie te verbeteren op elk level.
Zero Trust grootste focus is resource bescherming en on premise data bescherming. Toegang wordt nooit expliciet weergegeven en moet continue geëvalueerd worden.
De Trusted Internet Connections (TIC) leveren een sterke internet gateway. Dit helpt om aanvallen te blokkeren op het internet, maar hebben minder nut bij interne netwerken om dreigingen te detecteren en deze te blokkeren.
Hiervoor komt ZTA te pas. Het beschermen van de privacy van gebruikers en private informatie is een grote zorg in elke organisatie, en moet worden aangekaard. Privacy en data bescherming zijn te vinden in compliance programs zoals Federal Information Security Modernization Act (FISMA) en de Health Insurance Portability and Accountability Act (HIPAA).
Als antwoord hierop heeft NIST een framework ontworpen, namelijk, NISTPRIV. Deze framework bevat informatie omtrent hoe een bedrijf moet omgaan met privacy risico's en mitigatie strategieën, alsook hoe een bedrijf gevoelige data moet bewaren, beveiligen en identificeren.

Er zijn verschillende aanpakken mogelijk bij het opzetten van een Zero Trust Architecture. 

\begin{itemize}
    \item Pure Zero Trust Architecture
    \item Hybrid ZTA and Perimeter-Based Architecture
    \item ...
\end{itemize} 


