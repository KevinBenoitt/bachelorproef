%%=============================================================================
%% Inleiding
%%=============================================================================

\chapter{\IfLanguageName{dutch}{Inleiding}{Introduction}}%
\label{ch:inleiding}

Veel meer bedrijven en organisaties maken gebruik van Microsoft365 op grootschalig vlak maar ook op kleinschalig vlak.
Het komt vaker voor dat het gehele bedrijf op producten en services van Microsoft365 draaien. Hoe waarborg je de compliance, de informatie beveiliging en de efficiënte werking van Microsoft365?
Dit komt ter sprake in onderstaande verdeling en hoe het aangepakt kan worden.

\begin{itemize}
  \item probleemstelling
  \item onderzoeksdoelstelling
  \item onderzoeksvraag
  \item opzet van de bachelorproef
\end{itemize}

\section{\IfLanguageName{dutch}{Probleemstelling}{Problem Statement}}%
\label{sec:probleemstelling}

Zoals eerder vermeld, wordt Microsoft365 vaker geïmplementeerd in organisaties. Doordat dit zo een cruciale rol speelt in bedrijven, is de waarborg van informatie security en efficiëntie dat dit moet bieden cruciaal en moet dit optimaal zijn.
Veel Business Managers zouden de nodige standaarden en guidelines willen toepassen maar hebben geen inzicht of geen concreet plan hoe ze dit best aanpakken.
In dit onderzoek zal er aangekaart worden welke standaarden en policies best worden geïmplementeerd voor bedrijven die Microsoft365 gebruiken over hun gehele infrastructuur.
Dit wordt gedaan aan de hand van een onderzoek rond deze verschillende Governance Frameworks.


\section{\IfLanguageName{dutch}{Onderzoeksvraag}{Research question}}%
\label{sec:onderzoeksvraag}

Welke standaarden en governance frameworks zijn cruciaal om geïmplementeerd te worden in organisaties die over hun gehele infrastructuur gebruik maken van Microsoft 365 services en producten.
Hoe ga je hiermee te werk en wat zijn de stappen die je moet nemen om dit tot een goed einde te brengen?

\section{\IfLanguageName{dutch}{Onderzoeksdoelstelling}{Research objective}}%
\label{sec:onderzoeksdoelstelling}

Het beoogde resultaat van deze paper is een verduidelijking in de verschillende standaarden en governance frameworks die gelinkt zijn aan Microsoft 365 services en producten.
Een inzicht hebben over welke standaarden nodig zijn en welke optioneel zijn. Hoe je dit kan toepassen in je organisatie.
Het succes van deze onderzoek zal gemeten worden door een proof of concept met volgende aspecten:
\begin{itemize}
  \item Policy Implementation
  \item User managament
  \item Security \& Compliance
  \item Best Practices
\end{itemize}

Dit wordt dan vervolgd door een uitgebreid rapportageverslag.

\section{\IfLanguageName{dutch}{Opzet van deze bachelorproef}{Structure of this bachelor thesis}}%
\label{sec:opzet-bachelorproef}

% Het is gebruikelijk aan het einde van de inleiding een overzicht te
% geven van de opbouw van de rest van de tekst. Deze sectie bevat al een aanzet
% die je kan aanvullen/aanpassen in functie van je eigen tekst.

Dit is hoe de paper opgedeeld is:

In Hoofdstuk~\ref{ch:stand-van-zaken} worden de belangrijke technische termen nader uitgelegd en bekeken. Om een beter begrip te verkrijgen over het onderzoeksdomein.
Het kan voorkomen dat verschillende zaken die uitgelegd werden in de literatuurstudie misschien niet aan bod komen, dit zal voornamelijk zijn als er gekeken wordt naar de Proof of Concept (PoC) en men hier geen toevoeging aan ziet.

In Hoofdstuk~\ref{ch:methodologie} zal er eerst sprake zijn van documentatieanalyse en literatuurstudie. In eerste instantie is er dus literatuurstudie waar er een beter begrip wordt toegesteld voor de effectieve PoC.
Vervolgens wordt er ook gekeken naar de documentatieanalyse en zal er dus een onderzoek zijn naar eerdere case studies omtrent de governance frameworks.
Uiteindelijk volgt er een Proof of Concept die de policy implementation, user managament, security \& compliance en best practices gaat onderzoeken.

Met als laatste een uitgebreid rapportageverslag over de gemeten resultaten van the proof of concept.

% TODO: Vul hier aan voor je eigen hoofstukken, één of twee zinnen per hoofdstuk

In Hoofdstuk~\ref{ch:conclusie}, tenslotte, heeft dit onderzoek  als doel om mensen in het vakgebied van governance frameworks in relatie met M365-implementaties te voorzien van een duidelijke documentatie wat deze frameworks precies inhouden en hoe dit kan toegepast worden in IT-infrastructuren. Wat de voordelen, impact \& effeciëntie hiervan is te verantwoorden. Het verwachte doel is dat er een realistisch beeld komt van de werkwijze en richtlijnen om de implementatie van deze frameworks te realiseren in een IT organisatie waar M365-implementaties centraal staan.
Het onderzoek zal een zekere bijdrage leveren voor organisaties die meer informatie vergen te trachten over IT Governance en Frameworks en de bijhorende standaarden.
