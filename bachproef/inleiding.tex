%%=============================================================================
%% Inleiding
%%=============================================================================

\chapter{\IfLanguageName{dutch}{Inleiding}{Introduction}}%
\label{ch:inleiding}

Veel meer bedrijven en organisaties maken gebruik van Microsoft 365 (M365) op grootschalig vlak maar ook op kleinschalig vlak.
Het komt vaker voor dat het gehele bedrijf op producten en services van Microsoft365 draaien. Hoe waarborg je de informatie beveiliging en de efficiënte werking van Microsoft 365?
Dit komt ter sprake in onderstaande verdeling en hoe het aangepakt kan worden.


\begin{itemize}
  \item probleemstelling
  \item onderzoeksdoelstelling
  \item onderzoeksvraag
  \item opzet van de bachelorproef
\end{itemize}

\section{\IfLanguageName{dutch}{Probleemstelling}{Problem Statement}}%
\label{sec:probleemstelling}

Veel Business Managers zouden de nodige implementaties en standaarden willen toepassen maar hebben geen inzicht of geen concreet plan hoe ze dit best kunnen aanpakken.
In dit onderzoek zal er aangekaart worden welke implementaties \& standaarden er noodzakelijk zijn om de informatiebeheer te beschermen voor bedrijven die M365 gebruiken over hun gehele infrastructuur.
Dit wordt gedaan aan de hand van een onderzoek rond Microsoft Purview en bijhorende M365 tools.


\section{\IfLanguageName{dutch}{Onderzoeksvraag}{Research question}}%
\label{sec:onderzoeksvraag}


Hoe beïnvloedt de implementatie van Microsoft 365, met specifieke aandacht voor Microsoft Purview, DLP, Sensitivity Labels en Retention Labels, de LifeCycle en het beheer van documenten en data binnen organisaties?

\section{\IfLanguageName{dutch}{Onderzoeksdoelstelling}{Research objective}}%
\label{sec:onderzoeksdoelstelling}

Het beoogde resultaat van deze paper is een verduidelijking in de mogelijkheden van Informatiebeheer in een organisatie die M365 gebruiken over hun gehele infrastructuur.
Een inzicht hebben welke M365 Tools \& standaarden noodzakelijk zijn voor een LifeCycle policy binnen een organisatie.
Het succes van deze onderzoek zal gemeten worden door een Proof of Concept (PoC) met volgende aspecten:
\begin{itemize}
  \item Doelstelling van de PoC
  \item M365 Tools implementatie
  \item Scenario's uitschrijven en Testen
  \item Best Practices
\end{itemize}

Dit wordt dan vervolgd door een uitgebreid rapportageverslag.

\section{\IfLanguageName{dutch}{Opzet van deze bachelorproef}{Structure of this bachelor thesis}}%
\label{sec:opzet-bachelorproef}

% Het is gebruikelijk aan het einde van de inleiding een overzicht te
% geven van de opbouw van de rest van de tekst. Deze sectie bevat al een aanzet
% die je kan aanvullen/aanpassen in functie van je eigen tekst.

Dit is hoe de paper opgedeeld is:

In Hoofdstuk~\ref{ch:stand-van-zaken} worden de belangrijke technische termen nader uitgelegd en bekeken. Om een beter begrip te verkrijgen over het onderzoeksdomein.
Het kan voorkomen dat verschillende zaken die uitgelegd werden in de literatuurstudie misschien niet aan bod komen, dit zal voornamelijk zijn als er gekeken wordt naar de Proof of Concept (PoC) en men hier geen toevoeging aan ziet.

In Hoofdstuk~\ref{ch:methodologie} zal er eerst sprake zijn van documentatieanalyse en literatuurstudie. In eerste instantie is er dus literatuurstudie waar er een beter begrip wordt toegesteld voor de effectieve PoC.
Vervolgens wordt er ook gekeken naar de documentatieanalyse en zal er dus een onderzoek zijn naar eerdere case studies omtrent de LifeCycle documentatie.
Uiteindelijk volgt er een Proof of Concept die de doelstelling van de PoC, M365 Tools, Scenario testen en best practices gaat onderozkeen.

Met als laatste een uitgebreid rapportageverslag over de gemeten resultaten van the proof of concept.

% TODO: Vul hier aan voor je eigen hoofstukken, één of twee zinnen per hoofdstuk


In Hoofdstuk~\ref{ch:conclusie}, tenslotte, heeft dit onderozek als doel om mensen in het vakgebied van Microsoft Purview en Document LifeCycle te voorzien van een duidelijke documentatie wat LifeCycle documenten en data precies inhouden en hoe dit kan toegepast worden in IT-infrastructuren. Wat de voordelen, impact en effeciëntie hiervan is te verantwoorden. Het verwachte doel is dat er een realistisch beeld komt van de werkwijze en mogelijkheden binnen M365 om een levencyclus van documenten te realiseren in een IT organisatie waar M365-implementaties centraal staan.
Het onderzoek zal een zekere bijdrage leveren voor organisaties die meer informatie vergen te trachten over het opzetten van een levencyclus voor documenten en data.

