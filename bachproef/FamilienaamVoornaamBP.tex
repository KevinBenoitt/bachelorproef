%===============================================================================
% LaTeX sjabloon voor de bachelorproef toegepaste informatica aan HOGENT
% Meer info op https://github.com/HoGentTIN/latex-hogent-report
%===============================================================================

\documentclass[dutch,dit,thesis]{hogentreport}

% TODO:
% - If necessary, replace the option `dit`' with your own department!
%   Valid entries are dbo, dbt, dgz, dit, dlo, dog, dsa, soa
% - If you write your thesis in English (remark: only possible after getting
%   explicit approval!), remove the option "dutch," or replace with "english".

\usepackage{lipsum} % For blind text, can be removed after adding actual content

%% Pictures to include in the text can be put in the graphics/ folder
\graphicspath{{graphics/}}

%% For source code highlighting, requires pygments to be installed
%% Compile with the -shell-escape flag!
\usepackage[section]{minted}
%% If you compile with the make_thesis.{bat,sh} script, use the following
%% import instead:
%% \usepackage[section,outputdir=../output]{minted}
\usemintedstyle{solarized-light}
\definecolor{bg}{RGB}{253,246,227} %% Set the background color of the codeframe

%% Change this line to edit the line numbering style:
\renewcommand{\theFancyVerbLine}{\ttfamily\scriptsize\arabic{FancyVerbLine}}

%% Macro definition to load external java source files with \javacode{filename}:
\newmintedfile[javacode]{java}{
    bgcolor=bg,
    fontfamily=tt,
    linenos=true,
    numberblanklines=true,
    numbersep=5pt,
    gobble=0,
    framesep=2mm,
    funcnamehighlighting=true,
    tabsize=4,
    obeytabs=false,
    breaklines=true,
    mathescape=false
    samepage=false,
    showspaces=false,
    showtabs =false,
    texcl=false,
}

% Other packages not already included can be imported here

%%---------- Document metadata -------------------------------------------------
% TODO: Replace this with your own information
\author{Ernst Aarden}
\supervisor{Dhr. F. Van Houte}
\cosupervisor{Mevr. S. Beeckman}
\title[Optionele ondertitel]%
    {Titel van de bachelorproef}
\academicyear{\advance\year by -1 \the\year--\advance\year by 1 \the\year}
\examperiod{1}
\degreesought{\IfLanguageName{dutch}{Professionele bachelor in de toegepaste informatica}{Bachelor of applied computer science}}
\partialthesis{false} %% To display 'in partial fulfilment'
%\institution{Internshipcompany BVBA.}

%% Add global exceptions to the hyphenation here
\hyphenation{back-slash}

%% The bibliography (style and settings are  found in hogentthesis.cls)
\addbibresource{bachproef.bib}            %% Bibliography file
\addbibresource{../voorstel/voorstel.bib} %% Bibliography research proposal
\defbibheading{bibempty}{}

%% Prevent empty pages for right-handed chapter starts in twoside mode
\renewcommand{\cleardoublepage}{\clearpage}

\renewcommand{\arraystretch}{1.2}

%% Content starts here.
\begin{document}

%---------- Front matter -------------------------------------------------------

\frontmatter

\hypersetup{pageanchor=false} %% Disable page numbering references
%% Render a Dutch outer title page if the main language is English
\IfLanguageName{english}{%
    %% If necessary, information can be changed here
    \degreesought{Professionele Bachelor toegepaste informatica}%
    \begin{otherlanguage}{dutch}%
       \maketitle%
    \end{otherlanguage}%
}{}

%% Generates title page content
\maketitle
\hypersetup{pageanchor=true}

%%=============================================================================
%% Voorwoord
%%=============================================================================

\chapter*{\IfLanguageName{dutch}{Woord vooraf}{Preface}}%
\label{ch:voorwoord}

%% TODO:
%% Het voorwoord is het enige deel van de bachelorproef waar je vanuit je
%% eigen standpunt (``ik-vorm'') mag schrijven. Je kan hier bv. motiveren
%% waarom jij het onderwerp wil bespreken.
%% Vergeet ook niet te bedanken wie je geholpen/gesteund/... heeft

\lipsum[1-2]
%%=============================================================================
%% Samenvatting
%%=============================================================================

% TODO: De "abstract" of samenvatting is een kernachtige (~ 1 blz. voor een
% thesis) synthese van het document.
%
% Een goede abstract biedt een kernachtig antwoord op volgende vragen:
%
% 1. Waarover gaat de bachelorproef?
% 2. Waarom heb je er over geschreven?
% 3. Hoe heb je het onderzoek uitgevoerd?
% 4. Wat waren de resultaten? Wat blijkt uit je onderzoek?
% 5. Wat betekenen je resultaten? Wat is de relevantie voor het werkveld?
%
% Daarom bestaat een abstract uit volgende componenten:
%
% - inleiding + kaderen thema
% - probleemstelling
% - (centrale) onderzoeksvraag
% - onderzoeksdoelstelling
% - methodologie
% - resultaten (beperk tot de belangrijkste, relevant voor de onderzoeksvraag)
% - conclusies, aanbevelingen, beperkingen
%
% LET OP! Een samenvatting is GEEN voorwoord!

%%---------- Nederlandse samenvatting -----------------------------------------
%
% TODO: Als je je bachelorproef in het Engels schrijft, moet je eerst een
% Nederlandse samenvatting invoegen. Haal daarvoor onderstaande code uit
% commentaar.
% Wie zijn bachelorproef in het Nederlands schrijft, kan dit negeren, de inhoud
% wordt niet in het document ingevoegd.

\IfLanguageName{english}{%
\selectlanguage{dutch}
\chapter*{Samenvatting}
\lipsum[1-4]
\selectlanguage{english}
}{}

%%---------- Samenvatting -----------------------------------------------------
% De samenvatting in de hoofdtaal van het document

\chapter*{\IfLanguageName{dutch}{Samenvatting}{Abstract}}

\lipsum[1-4]


%---------- Inhoud, lijst figuren, ... -----------------------------------------

\tableofcontents

% In a list of figures, the complete caption will be included. To prevent this,
% ALWAYS add a short description in the caption!
%
%  \caption[short description]{elaborate description}
%
% If you do, only the short description will be used in the list of figures

\listoffigures

% If you included tables and/or source code listings, uncomment the appropriate
% lines.
%\listoftables
%\listoflistings

% Als je een lijst van afkortingen of termen wil toevoegen, dan hoort die
% hier thuis. Gebruik bijvoorbeeld de ``glossaries'' package.
% https://www.overleaf.com/learn/latex/Glossaries

%---------- Kern ---------------------------------------------------------------

\mainmatter{}

% De eerste hoofdstukken van een bachelorproef zijn meestal een inleiding op
% het onderwerp, literatuurstudie en verantwoording methodologie.
% Aarzel niet om een meer beschrijvende titel aan deze hoofdstukken te geven of
% om bijvoorbeeld de inleiding en/of stand van zaken over meerdere hoofdstukken
% te verspreiden!

%%=============================================================================
%% Inleiding
%%=============================================================================

\chapter{\IfLanguageName{dutch}{Inleiding}{Introduction}}%
\label{ch:inleiding}

Veel meer bedrijven en organisaties maken gebruik van Microsoft 365 (M365) op grootschalig vlak maar ook op kleinschalig vlak.
Het komt vaker voor dat het gehele bedrijf op producten en services van Microsoft365 draaien. Hoe waarborg je de informatie beveiliging en de efficiënte werking van Microsoft 365?
Dit komt ter sprake in onderstaande verdeling en hoe het aangepakt kan worden.


\begin{itemize}
  \item probleemstelling
  \item onderzoeksdoelstelling
  \item onderzoeksvraag
  \item opzet van de bachelorproef
\end{itemize}

\section{\IfLanguageName{dutch}{Probleemstelling}{Problem Statement}}%
\label{sec:probleemstelling}

Veel Business Managers zouden de nodige implementaties en standaarden willen toepassen maar hebben geen inzicht of geen concreet plan hoe ze dit best kunnen aanpakken.
In dit onderzoek zal er aangekaart worden welke implementaties \& standaarden er noodzakelijk zijn om de informatiebeheer te beschermen voor bedrijven die M365 gebruiken over hun gehele infrastructuur.
Dit wordt gedaan aan de hand van een onderzoek rond Microsoft Purview en bijhorende M365 tools.


\section{\IfLanguageName{dutch}{Onderzoeksvraag}{Research question}}%
\label{sec:onderzoeksvraag}


Hoe beïnvloedt de implementatie van Microsoft 365, met specifieke aandacht voor Microsoft Purview, DLP, Sensitivity Labels en Retention Labels, de LifeCycle en het beheer van documenten en data binnen organisaties?

\section{\IfLanguageName{dutch}{Onderzoeksdoelstelling}{Research objective}}%
\label{sec:onderzoeksdoelstelling}

Het beoogde resultaat van deze paper is een verduidelijking in de mogelijkheden van Informatiebeheer in een organisatie die M365 gebruiken over hun gehele infrastructuur.
Een inzicht hebben welke M365 Tools \& standaarden noodzakelijk zijn voor een LifeCycle policy binnen een organisatie.
Het succes van deze onderzoek zal gemeten worden door een Proof of Concept (PoC) met volgende aspecten:
\begin{itemize}
  \item Doelstelling van de PoC
  \item M365 Tools implementatie
  \item Scenario's uitschrijven en Testen
  \item Best Practices
\end{itemize}

Dit wordt dan vervolgd door een uitgebreid rapportageverslag.

\section{\IfLanguageName{dutch}{Opzet van deze bachelorproef}{Structure of this bachelor thesis}}%
\label{sec:opzet-bachelorproef}

% Het is gebruikelijk aan het einde van de inleiding een overzicht te
% geven van de opbouw van de rest van de tekst. Deze sectie bevat al een aanzet
% die je kan aanvullen/aanpassen in functie van je eigen tekst.

Dit is hoe de paper opgedeeld is:

In Hoofdstuk~\ref{ch:stand-van-zaken} worden de belangrijke technische termen nader uitgelegd en bekeken. Om een beter begrip te verkrijgen over het onderzoeksdomein.
Het kan voorkomen dat verschillende zaken die uitgelegd werden in de literatuurstudie misschien niet aan bod komen, dit zal voornamelijk zijn als er gekeken wordt naar de Proof of Concept (PoC) en men hier geen toevoeging aan ziet.

In Hoofdstuk~\ref{ch:methodologie} zal er eerst sprake zijn van documentatieanalyse en literatuurstudie. In eerste instantie is er dus literatuurstudie waar er een beter begrip wordt toegesteld voor de effectieve PoC.
Vervolgens wordt er ook gekeken naar de documentatieanalyse en zal er dus een onderzoek zijn naar eerdere case studies omtrent de LifeCycle documentatie.
Uiteindelijk volgt er een Proof of Concept die de doelstelling van de PoC, M365 Tools, Scenario testen en best practices gaat onderozkeen.

Met als laatste een uitgebreid rapportageverslag over de gemeten resultaten van the proof of concept.

% TODO: Vul hier aan voor je eigen hoofstukken, één of twee zinnen per hoofdstuk


In Hoofdstuk~\ref{ch:conclusie}, tenslotte, heeft dit onderozek als doel om mensen in het vakgebied van Microsoft Purview en Document LifeCycle te voorzien van een duidelijke documentatie wat LifeCycle documenten en data precies inhouden en hoe dit kan toegepast worden in IT-infrastructuren. Wat de voordelen, impact en effeciëntie hiervan is te verantwoorden. Het verwachte doel is dat er een realistisch beeld komt van de werkwijze en mogelijkheden binnen M365 om een levencyclus van documenten te realiseren in een IT organisatie waar M365-implementaties centraal staan.
Het onderzoek zal een zekere bijdrage leveren voor organisaties die meer informatie vergen te trachten over het opzetten van een levencyclus voor documenten en data.


\chapter{\IfLanguageName{dutch}{Stand van zaken}{State of the art}}%
\label{ch:stand-van-zaken}

% Tip: Begin elk hoofdstuk met een paragraaf inleiding die beschrijft hoe
% dit hoofdstuk past binnen het geheel van de bachelorproef. Geef in het
% bijzonder aan wat de link is met het vorige en volgende hoofdstuk.

% Pas na deze inleidende paragraaf komt de eerste sectiehoofding.

 % Dit hoofdstuk bevat je literatuurstudie. De inhoud gaat verder op de inleiding, maar zal het onderwerp van de bachelorproef *diepgaand* uitspitten. De bedoeling is dat de lezer na lezing van dit hoofdstuk helemaal op de hoogte is van de huidige stand van zaken (state-of-the-art) in het onderzoeksdomein. Iemand die niet vertrouwd is met het onderwerp, weet nu voldoende om de rest van het verhaal te kunnen volgen, zonder dat die er nog andere informatie moet over opzoeken \autocite{Pollefliet2011}.

 %M Je verwijst bij elke bewering die je doet, vakterm die je introduceert, enz.\ naar je bronnen. In \LaTeX{} kan dat met het commando \texttt{$\backslash${textcite\{\}}} of \texttt{$\backslash${autocite\{\}}}. Als argument van het commando geef je de ``sleutel'' van een ``record'' in een bibliografische databank in het Bib\LaTeX{}-formaat (een tekstbestand). Als je expliciet naar de auteur verwijst in de zin (narratieve referentie), gebruik je \texttt{$\backslash${}textcite\{\}}. Soms is de auteursnaam niet expliciet een onderdeel van de zin, dan gebruik je \texttt{$\backslash${}autocite\{\}} (referentie tussen haakjes). Dit gebruik je bv.~bij een citaat, of om in het bijschrift van een overgenomen afbeelding, broncode, tabel, enz. te verwijzen naar de bron. In de volgende paragraaf een voorbeeld van elk.

 % \textcite{Knuth1998} schreef een van de standaardwerken over sorteer- en zoekalgoritmen. Experten zijn het erover eens dat cloud computing een interessante opportuniteit vormen, zowel voor gebruikers als voor dienstverleners op vlak van informatietechnologie~\autocite{Creeger2009}.

 %Let er ook op: het \texttt{cite}-commando voor de punt, dus binnen de zin. Je verwijst meteen naar een bron in de eerste zin die erop gebaseerd is, dus niet pas op het einde van een paragraaf.


Om de titel en onderzoek te verduidelijken, wordt er eerst onderzoek verricht naar verschillende relevante aspecten van het onderwerp. Onderdelen zoals Governance Frameworks, ISO, IEC, IEEE, NIST en COBIT komen ter sprake.



\subsection{Governane Frameworks}

De digitale wereld blijft groeien waardoor Information Technology een belangrijkere rol begint te spelen in organisaties; Specifieker, organisaties die Microsoft365 implementeren in hun infrastructuur.
IT governance is het proces waarbij beslissingen worden gemaakt rond IT investments. Elke organisatie heeft wel een level van IT governance verwerkt, maar vaak is dit niet volledig gestructureerd en ontbreken er policies en is het heel informeel. \autocite{Symons2005}
Microsoft365 Governance Frameworks zijn dus gericht op policies, guidelines en best practices voor een bedrijf die M365 implementeert. De bedoeling hiervan is het waarborgen van efficiëntie van M365 services maar terwijl ook de veiligheid en compliance van de organisatie te garanderen.


\subsection{Azure Cloud Governance}

Het basis principe van cloud computing is dat een bedrijf in plaats van eigen hardware aan te kopen en beheren, dat ze resources huren van een cloud provider. Met als voordeel dat je alleen betaalt, wat je verbruikt. \autocite{Mikkola2021}
De cloud computing market heeft een grote sprong gemaakt in de IT wereld tijdens de Covid-19 pandemie, de combinatie met IT en remote werk zorgde voor een grote sprong in deze markt.
Natuurlijk zijn er verschillende Cloud Computing Deployment Models:

\begin{itemize}
    \item Private cloud:
    Dit is een cloud infrastructuur gemaakt voor één organisatie en niet meer. Dit kan dus on-premise beheerd worden of remote.
    Een reden om voor Private cloud te kiezen, is dat een eenvoudige manier is om gevoelige data en informatie te beveiligen en isoleren. \autocite{Mikkola2021}
    \item Public cloud:
    Dit is de cloud infrastructuur gemaakt voor publiek gebruik en wordt beheerd on premise door de Cloud provider.
    \item Hybrid cloud:
    Dit is de combinatie van de 2 voorgaande deployment models. Verschillende onderdelen binnen deze infrastuctuur zijn privaat, waar anderen dan weer publiekelijk zijn. Dit hangt volledig af van de voorafgeschreven scenario's.
\end{itemize}

Natuurlijk vraagt men zich af wat de voordelen nu zijn van Cloud Computing. Door de immense groei van toename in gebruikers van cloud computing, geeft dit ook een teken dat hier voordelen aan gekoppeld zijn.

\begin{itemize}
    \item Scalability:
    De mogelijkheid hebben om je resources te scalen gebaseerd op gebruik, verbruik en vraag. \autocite{Mikkola2021}
    \item Security:
    De security van de cloud infrastructuur is up-to-date en wordt regelmatig geüpdate naargelang de meest recente security threats.
    \item Geen overnodige on-premise hardware:
    De overbodige on-premise hardware wordt vervangen door een cloud dat remote te besturen is, dit is financieel heel voordelig.
    \item Return of Investment (ROI):
    Doordat een organisatie zelf in handen hebben, hoeveel er verbruikt kan worden binnenin hun organisatie. Hebben ze ook meer in hand wat de effectieve ROI zou zijn. Dit biedt een vooruitzicht op financiele doeleinden.
\end{itemize}


\subsection{ISO/IEC}

De International Organization for Standardization (ISO) is een onafhankelijk wereldwijd bedrijf dat instaat voor de normen waar een organisatie aan moet voldoen. Zo kan een bedrijf de veiligheid en kwaliteit van hun services garanderen.
ISO werkt nauw samen met International Electrotechnical Commission (IEC) dat verantwoordelijk is voor de ontwikkeling van standaarden op het gebied van elektronische technologieën en verwante sectoren, waardoor uniforme normen en protocollen worden vastgesteld voor elektronica en aanverwante gebieden. \autocite{Florea2016}
Natuurlijk zijn er verschillende ISO protocollen die werden aangemaakt om zo de normen te bereiken dat een bedrijf eist:

\begin{itemize}
    \item ISO/IEC 38500:
    Dit is een internationale standaard voor corporate governance of information technology. Het doel van deze standaard is om de organisatie management te voorzien van principles, definitions en een deployment model zodat de organisatie de Information Technology kan evalueren, beheren en monitoren. \autocite{Mikkola2021}
    \item ISO/IEC 27001:
    Dit is een standaard met sprake op Information Systems Security Management (ISMS) en heeft betrekking op Information Security en het aanliggende risico dat dit teweeg brengt. Het einddoel is de ISMS te optimaliseren op elk aspect. \autocite{AlMayahi2012}
    \item ISO 31000:
    Dit heeft betrekking tot de Risk Management. Elke organisatie heeft wel een gradatie in hoe ze omgaan met Risk Management. Maar deze standaard heeft een talrijk aantal principles om de Risk Management te optimaliseren. \autocite{Florea2016}
    Deze standaard is niet specifiek gericht naar IT infrastructuren, maar kan wel een bijdrage leveren in elke organisatie. 
\end{itemize}


\subsection{ITIL}

ITIL werd origineel opgericht in de United Kingdom door de Central Computer and Telecommunications Agency (CCTA) in 1980 om de IT Service Management te verbeteren in de UK central government. \autocite{Potgieter2005}


De Information Technology Infrastructure Library (ITIL) is een wereldwijd gekend IT service management framework dat bestaat uit guidelines en best practices.  \autocite{Potgieter2005}
De framework zorgt ervoor dat er duidelijke structuur zit, om zo organisaties te helpen bij het leveren en beheren van hun IT en digital services. Om zo een een optimale value te leveren aan eindgebruikers, klanten en stakeholders. 
De ITIL framework laat de service provider toe om de services aan te passen naargelang de eisen, visie, strategieën en einddoel van een organisatie. \autocite{Mikkola2021}

De huidige versie van ITIL, ITIL 4, is gepubliceerd in Februari 2019. \autocite{Mikkola2021}
Natuurlijk vraagt men zich af, wat de relatie tot IT Governance dit brengt.De IT service management is gefocused op IT services en gerelateerde services.
Waar IT Governance dan vooral gefocust is op het enablen, controleren en assisteren bij het maken van beslissingen op strategisch niveau in organisaties. ITIL framework helpt hierbij op verschillende aspecten door het controleren en beheren van technologische veranderingen om zo het gebruik van IT services te verbeteren binnen het bedrijf. \autocite{Mikkola2021}

\subsection{COBIT}
Control Objectives for Information and Related Technologies is ontwikkeld door Information Systems Audit and Control Association (ISACA) en is een IT Governance Framework. Het doel van deze framework is om goede IT governance, technology management en de eventuele risk management te combineren. \autocite{Khther2013} 
Het is onafgebroken bezig met het updaten van policies en guidelines, maar werkt ook harmonisch samen met andere standaarden en guidelines. COBIT framework assisteert organisaties in de huidige challenges aan te pakken op vlak van de business arena door te relateren naar de requirements van businessen.
De meest recente versie van COBIT is COBIT 2019. De voorganger hiervan was COBIT 5, door gebrek aan support voor nieuwe technologie binnen de IT was de vraag groot naar een nieuwere versie.

Volgens ISACA biedt COBIT 2019 de volgende verbeteringen aan:

\begin{itemize}
    \item Flexibility
    Door de introductie van design factors kon COBIT 2019 flexibeler zijn en opener toe naar organisaties en gebruikers. Dit liet COBIT toe om aangepast te zijn aan specifieke user contexten. De COBIT open architecture geeft de toevoeging om meer te focussen op nieuwe areas en de modificaties op bestaane modellen zonder een impact teweeg te brengen op de core model van COBIT. \autocite{Mikkola2021}
    \item Relevantie
    Het COBIT Model ondersteunt alignment naar concepten toe die oorspronkelijk afstammen van een externe sources met name de laatste IT standaarden en compliance regulaties. \autocite{Khther2013}
    \item Performance Management of IT 
    De structuur van COBIT 2019 perfomance management model is geïntegreerd in het conceptuele model.
    \item Efficiëntie
    De efficiëntie van hoe information moet verwerkt worden op de meest optimale gebruik van resources.
    \item Confidentiality
    De bescherming van gevoelige informatie voor onbevoegde toegangsbeheer.
    \item Compliance
    De compliance met de wet, regulaties en contractuele arrangementen waar een business onderdeel van is.
\end{itemize}

Een onderdeel van COBIT is de COBIT Control Objectives. Deze bestaan om te assisteren in het bouwen van een pasbare management en control system binnen een IT omgeving. \autocite{Khther2013} 
Om ervoor te zorgen dat er een continue service voorzien is, dat gevestigd kan worden door de implementatie van meerdere controle procedures zoals het testen en schrijven van continuity plans. \autocite{Khther2013} 


\subsection{NIST}
National Institute of Standards and Technology (NIST) is opgericht door de Federal Chief Information Officier (CIO) Vivek Kundra om de federale overheid hun security adoptie te doen verbeteren en versnellen in verband met Cloud Computing.
Dit wordt gedaan door een talrijk aantal standaarden en guidelines op te richten in dicht verband met stakeholders, grote en private bedrijven. \autocite{Hogan2011}
Dit is ontstaan door voorgaande cyber incidenten. In Maart 2018 heeft de FBI een alert uitgestuurd omtrent een hevige groei in criminele activiteiten, vooral gericht op organisaties binnen de Energie sector.
Een meer recente cybersecurity aanval was die van in 2017 bij het bedrijf Equifax, criminele hackers hadden data gebreached van over de 147 millioen mensen over de hele wereld door security lekken en bad security practises uit te buiten. Dit heeft het bedrijf meer dan 240 millioen gekost. \autocite{Calder2018}

\begin{itemize}
     \item Cybersecurity en Information Technology
     NIST speelt een cruciale rol in de ontwikkeling van guidelines en standaarden omtrent Cybersecurity. Dit heet de NIST Cybersecurity Framework (CSF).
     Dit framework heeft 5 steeds voorkomende functies: Identify, Protect, Detect, Respond and Recover. \autocite{Ibrahim2018}
     Als deze 5 functies collectief worden geïmplementeerd in een bedrijf, geeft dit een goed beeld weer van hoe een high-end security standaard er zou moeten uitzien.
     Dit dient als een leiddraad voor bedrijven of organisaties om hun cybersecurity strategieplan te ontwikkelen en te verbeteren. \autocite{Almuhammadi2017}
     De framework biedt dus een gemeenschappelijke taalbegrip voor zowel interne als externe stakeholders. Het kan gebruikt worden voor:

     \begin{itemize}
        \item Het identificeren en prioritiseren van acties om de cybersecurity risk te verlagen
        \item Is een tool om de business, policy en technologische benadering te laten afstemmen met elkaar. \autocite{Calder2018}
        \item Om risico's te beheren voor de gehele organisatie.
        \item Om te focussen op kritische aspecten binnen een organisatie.
     \end{itemize}

    \item International
    NIST heeft een grote impact op de internationale gemeenschap rond beveiliging en standaarden. Veel organisaties hun beleid is gebaseerd op de NIST richtlijnen en gebruiken dit als referentiekader.
    Niet alleen de U.S. maakt gebruik van NIST. Landen zoals Italië en Uruguay hebben de framework reeds geïmplementeerd en andere landen hebben hun eigen variatie met NIST als hun referentiekader opgesteld. \autocite{Calder2018}
\end{itemize}

Om een beter begrip te krijgen hoe je data en informatie beveiligd moet je weten wat beveiiging eigenlijk betekent. Informatie en data moeten beveiligd volgens de CIA-driehoek: \autocite{Calder2018}
\begin{itemize}
    \item Confidentiality
    Informatie en data mogen alleen beschikbaar zijn voor mensen die het toegang verachten.
    \item Integrity
    Informatie en data zouden beschermd moeten zijn tegen data verlies of data verwijdering.
    \item Availability
    Informatie en data beschikbaarheid alleen voor bevoegden en bepaalde tijdstippen
\end{itemize}

\subsection{Cybersecurity Framework (CSF)}

Doorgaans is CSF implementeerbaar in elk soort organisatie. Maar is vooral gefocused op organisatie met kritieke data en informatie en een hoog risico lopen voor hackers.
De framework is een lopend document dat altijd updates en verbeteringen zal toevoegen met feedback van reeds bestaande implementaties. \autocite{Calder2018}

%%=============================================================================
%% Methodologie
%%=============================================================================

\chapter{\IfLanguageName{dutch}{Methodologie}{Methodology}}%
\label{ch:methodologie}

%% TODO: In dit hoofstuk geef je een korte toelichting over hoe je te werk bent
%% gegaan. Verdeel je onderzoek in grote fasen, en licht in elke fase toe wat
%% de doelstelling was, welke deliverables daar uit gekomen zijn, en welke
%% onderzoeksmethoden je daarbij toegepast hebt. Verantwoord waarom je
%% op deze manier te werk gegaan bent.
%% 
%% Voorbeelden van zulke fasen zijn: literatuurstudie, opstellen van een
%% requirements-analyse, opstellen long-list (bij vergelijkende studie),
%% selectie van geschikte tools (bij vergelijkende studie, "short-list"),
%% opzetten testopstelling/PoC, uitvoeren testen en verzamelen
%% van resultaten, analyse van resultaten, ...
%%
%% !!!!! LET OP !!!!!
%%
%% Het is uitdrukkelijk NIET de bedoeling dat je het grootste deel van de corpus
%% van je bachelorproef in dit hoofstuk verwerkt! Dit hoofdstuk is eerder een
%% kort overzicht van je plan van aanpak.
%%
%% Maak voor elke fase (behalve het literatuuronderzoek) een NIEUW HOOFDSTUK aan
%% en geef het een gepaste titel.

\lipsum[21-25]



% Voeg hier je eigen hoofdstukken toe die de ``corpus'' van je bachelorproef
% vormen. De structuur en titels hangen af van je eigen onderzoek. Je kan bv.
% elke fase in je onderzoek in een apart hoofdstuk bespreken.

%\input{...}
%\input{...}
%...

%%=============================================================================
%% Conclusie
%%=============================================================================

\chapter{Conclusie}%
\label{ch:conclusie}

% TODO: Trek een duidelijke conclusie, in de vorm van een antwoord op de
% onderzoeksvra(a)g(en). Wat was jouw bijdrage aan het onderzoeksdomein en
% hoe biedt dit meerwaarde aan het vakgebied/doelgroep? 
% Reflecteer kritisch over het resultaat. In Engelse teksten wordt deze sectie
% ``Discussion'' genoemd. Had je deze uitkomst verwacht? Zijn er zaken die nog
% niet duidelijk zijn?
% Heeft het onderzoek geleid tot nieuwe vragen die uitnodigen tot verder 
%onderzoek?

\lipsum[76-80]



%---------- Bijlagen -----------------------------------------------------------

\appendix

\chapter{Onderzoeksvoorstel}

Het onderwerp van deze bachelorproef is gebaseerd op een onderzoeksvoorstel dat vooraf werd beoordeeld door de promotor. Dat voorstel is opgenomen in deze bijlage.

%% TODO: 
%\section*{Samenvatting}

% Kopieer en plak hier de samenvatting (abstract) van je onderzoeksvoorstel.

% Verwijzing naar het bestand met de inhoud van het onderzoeksvoorstel
%---------- Inleiding ---------------------------------------------------------

\section{Introductie}%
\label{sec:introductie}

\subsection{Context}
Steeds meer organisaties maken de overstap naar M365 als centraal onderdeel van hun infrastructuur. Een gebrek aan duidelijke regelgeving en documentatie kan leiden tot beveiligingsrisico's, inconsistenties en verkeerd geïmplementeerde frameworks. Hierdoor is de behoefte aan duidelijk gestructureerde documentatie met betrekking tot Governance Frameworks een noodzaak.

\subsection{Onderzoek}
Het belang van Governance Frameworks binnen IT-omgevingen neemt steeds meer toe, vooral voor organisaties waar M365-implementaties centraal staan. Een geoptimaliseerde en beveiligde infrastructuur mag hierbij niet ontbreken. Dit onderzoeksvoorstel omvat een grondige analyse van specifieke Governance Frameworks, met de nadruk op NIST, ISO \& COBIT. 
Het doel van dit onderzoek is om antwoorden te vinden met betrekking tot de impact, toepasbaarheid en de voordelen die de frameworks zullen bieden. Ook wordt er gekeken naar welke uitdagingen en complicaties zich voordoen en hoe deze aangepakt moeten worden om een geoptimaliseerde infrastructuur te creëren. Ook zal

\subsection{Doelstelling}
De doelstelling van dit onderzoek is om diepgaande kennis en inzicht te verkrijgen over de verschillende Governance Frameworks, met name NIST, ISO \& COBIT. Op basis hiervan zullen richtlijnen en adviezen worden opgesteld voor de implementatie binnen IT-organisaties die gericht zijn op M365.

\subsection{Doelgroep}
De doelgroep voor dit onderzoek richt zich vooral op IT-consultants met een specialisatie in M365, informatiebeheerders en IT-professionals die verantwoordelijk zijn voor de werkwijze en implementatie van Governance-structuren. 

\subsection{Conclusie \& Eindresultaat}
Het resultaat van dit onderzoek zal afhangen van de toestandskoming van de documentatie of rapport betreft Governance frameworks binnen IT-omgevingen gericht op M365-implementaties. Hiervoor zal er dus ook een Proof of Concept (PoC) opgesteld worden om de frameworks te realiseren in een M365 tenant omgeving.


%---------- Stand van zaken ---------------------------------------------------

\section{Literatuurstudie}%
\label{sec:Literatuurstudie}

Er is een exponentiële groei van cloudgebaseerde oplossingen zoals M365, dit wordt steeds meer een volstrekt onderdeel binnen organisatieinfrastructuren. M365 biedt een veelheid aan tools die de coöperatie, effeciëntie \& productiviteit verhogen. Ook speelt de IT Governance een cruciale rol binnen organisaties. 
De compliance-vereisten en beveiligsstandaarden is een directe uitnodiging om Governance Frameworks te implementeren binnen organisaties die gericht zijn op M365.
In deze literatuurstudie gaan we nader ingaan op deze frameworks met name National Institute of Standards (NIST), International Organization of Standardization (ISO) \& Control Objectives for Information and Related Technologies (COBIT). Het onderzoek zal een analyse van voorgaande frameworks over de rol, toepasbaarheid of functionaliteit \& invloed binnen een organisatie.

\subsection{Governance Framework}
Governance frameworks verwijzen naar het geheel van processen, beleid en verantwoordelijkheden om de goede werking, privacy en beveiliging van IT-systemen binnen organisaties te garanderen. Het moet ervoor zorgen dat de werking zo effeciënt, risicovrij en regelgeving te waarborgen. Maar voornamelijk de vereisten en noden van de IT-organisatie te voldoen. Dit zijn de belangrijkste aspecten binnenin IT-Governance:

\begin{itemize}
  \item Risk Management
  Hier is het van belang om eventuele gevaren te identificeren, te bestuderen en op te lossen in verband met verschillende IT-processen
  \item Performance Management
  Dit is belangrijk om IT processen te evalueren en knelpunten te identificeren om zo verbetersmogelijkheden te vinden en een effectievere werkwijze op te stellen.
  \item Resource Management
  Het beheer van de IT resources, hier vallen ook de financiën, personeel \& infrastructuur in. Dit geeft de mogelijkheid aan de verantwoordelijken en bevoegden om ervoor te zorgen dat er continue ondersteuning beschikbaar is voor de huidige en toekomstige (onvoorziene) kosten en investeringen.
  \item Strategy
  Een grote portie van elke IT Governance implementatie vergt veel aandacht aan de strategie. Technologische vooruitgang is blijvende groeiend organisme. Een strategieplan bevat de prioriteiten die moeten uitgevoerd worden, een reeks aan initiatieven dat moeten gebeuren bij specifieke scenarios. Ook worden er Disaster Recovery Plans (DRP) uitgeschreven indien er een breuk in de infrastructuur is.
  \item Compliance \& Regulations
  Dit zijn de vereisten \& regelgevingen met betrekking tot privacy, beveiliging \& gegevensbescherming.
\end{itemize}

% Voor literatuurverwijzingen zijn er twee belangrijke commando's:
% \autocite{KEY} => (Auteur, jaartal) Gebruik dit als de naam van de auteur
%   geen onderdeel is van de zin.
% \textcite{KEY} => Auteur (jaartal)  Gebruik dit als de auteursnaam wel een
%   functie heeft in de zin (bv. ``Uit onderzoek door Doll & Hill (1954) bleek
%   ...'')



%---------- Methodologie ------------------------------------------------------
\section{Methodologie}%
\label{sec:methodologie}

Hier beschrijf je hoe je van plan bent het onderzoek te voeren. Welke onderzoekstechniek ga je toepassen om elk van je onderzoeksvragen te beantwoorden? Gebruik je hiervoor literatuurstudie, interviews met belanghebbenden (bv.~voor requirements-analyse), experimenten, simulaties, vergelijkende studie, risico-analyse, PoC, \ldots?

Valt je onderwerp onder één van de typische soorten bachelorproeven die besproken zijn in de lessen Research Methods (bv.\ vergelijkende studie of risico-analyse)? Zorg er dan ook voor dat we duidelijk de verschillende stappen terug vinden die we verwachten in dit soort onderzoek!

Vermijd onderzoekstechnieken die geen objectieve, meetbare resultaten kunnen opleveren. Enquêtes, bijvoorbeeld, zijn voor een bachelorproef informatica meestal \textbf{niet geschikt}. De antwoorden zijn eerder meningen dan feiten en in de praktijk blijkt het ook bijzonder moeilijk om voldoende respondenten te vinden. Studenten die een enquête willen voeren, hebben meestal ook geen goede definitie van de populatie, waardoor ook niet kan aangetoond worden dat eventuele resultaten representatief zijn.

Uit dit onderdeel moet duidelijk naar voor komen dat je bachelorproef ook technisch voldoen\-de diepgang zal bevatten. Het zou niet kloppen als een bachelorproef informatica ook door bv.\ een student marketing zou kunnen uitgevoerd worden.

Je beschrijft ook al welke tools (hardware, software, diensten, \ldots) je denkt hiervoor te gebruiken of te ontwikkelen.

Probeer ook een tijdschatting te maken. Hoe lang zal je met elke fase van je onderzoek bezig zijn en wat zijn de concrete \emph{deliverables} in elke fase?

%---------- Verwachte resultaten ----------------------------------------------
\section{Verwacht resultaat, conclusie}%
\label{sec:verwachte_resultaten}

Hier beschrijf je welke resultaten je verwacht. Als je metingen en simulaties uitvoert, kan je hier al mock-ups maken van de grafieken samen met de verwachte conclusies. Benoem zeker al je assen en de onderdelen van de grafiek die je gaat gebruiken. Dit zorgt ervoor dat je concreet weet welk soort data je moet verzamelen en hoe je die moet meten.

Wat heeft de doelgroep van je onderzoek aan het resultaat? Op welke manier zorgt jouw bachelorproef voor een meerwaarde?

Hier beschrijf je wat je verwacht uit je onderzoek, met de motivatie waarom. Het is \textbf{niet} erg indien uit je onderzoek andere resultaten en conclusies vloeien dan dat je hier beschrijft: het is dan juist interessant om te onderzoeken waarom jouw hypothesen niet overeenkomen met de resultaten.



%%---------- Andere bijlagen --------------------------------------------------
% TODO: Voeg hier eventuele andere bijlagen toe. Bv. als je deze BP voor de
% tweede keer indient, een overzicht van de verbeteringen t.o.v. het origineel.
%\input{...}

%%---------- Backmatter, referentielijst ---------------------------------------

\backmatter{}


\setlength\bibitemsep{2pt} %% Add Some space between the bibliograpy entries
\printbibliography[heading=bibintoc]

\end{document}
